\documentclass{beamer}

\usepackage[czech]{babel}

\usepackage[T1]{fontenc}
\usepackage[utf8]{inputenc}
\usepackage{times}
\usepackage{hyperref}
\usepackage{caption}
\usepackage{listings}
\usepackage{minted}
%%\def\lcm{\mathop{\operator@font lcm}}
\newcommand{\nequiv}{\equiv \hspace{-3.5mm} / \hspace{2.5mm} }

\mode<presentation>{
\usetheme[headheight=5mm,footheight=5mm]{Madrid}
\usecolortheme{crane}
%\usetheme{Boadilla}
%\usecolortheme{seahorse}
%\useoutertheme{shadow}
%\usecolortheme{wolverine}
%\usecolortheme{beetle}
}

\def\m#1{{\color{blue} #1}}
%% TZ?
\def\rememberenum{\xdef\rememberenum{\@arabic\c@enumi}}
\def\continueenum{\c@enumi=\rememberenum}
%%%%%%%
%%%%%%%
\newcommand{\bi}{\begin{itemize}}%Definition des Befehls \bi
\newcommand{\ei}{\end{itemize}}%Definition des Befehls \ei
\newcommand{\be}{\begin{enumerate}}%Definition des Befehls \be
\newcommand{\ee}{\end{enumerate}}%Definition des Befehls \ee
%%%%%%%%%%%%%%%%%%%%%%%%%%%%%%%%%%%%%%%%%%%
%------------------------------------------------------
%  Následujících 4 řádky upravit
%------------------------------------------------------
\title[CVE-2019-0887]{Síťová a systémová bezpečnost}
    \subtitle{Windows reverse RDP attack\\CVE-2019-0887}
\author[J. Troják]{Jan Troják}
\date[BI-SSB, 2021]{} % Kód předmětu, datum, zkráceno "Přednáška číslo."
%------------------------------------------------------
%  Na tuto sekci nesahat
%------------------------------------------------------
\institute[ČVUT FIT]{%
\includegraphics[height=.12\textheight]{pic/LogoCVUT.pdf}\\ \smallskip
České vysoké učení technické v Praze, 
Fakulta informačních technologií\\
Katedra informační bezpečnosti\\ \smallskip
\tiny{
Tato prezentace byla vytvořena jako semestrální práce v předmětu BI-SSB
}
}
\begin{document}
\begin{frame}
\titlepage
\end{frame}

%%%%%%%%%%%%%%%%%%%%%%%%%%%%%%%%%%%%%%%%%%%%%%%%%%%%%%%%%%%%%%%%%%%%%%%%%%%%
%%%%%%%%%%%%%%%%%%%%%%%%%%%%%%%%%%%%%%%%%%%%%%%%%%%%%%%%%%%%%%%%%%%%%%%%%%%%
\begin{frame}
\frametitle{\color{red!40!blue}Remote Desktop Protocol (RDP)}
\tableofcontents
    \bi
        \item proprietární protokol pro vzdálenou správu počítače
        \item emuluje přístup ke GUI počítače
        \item vyvinut primárně pro Windows
        \item klient-server architektura
        \item hojně využívaný v IT podporách
        \item hojně využívaný při správě serverů s GUI
        \item defaultně komunikuje na portu 3389
        \item na Windows se používá v aplikaci \textit{Remote Desktop Connection} 
    \ei

\end{frame}
%---------------------------------------------------------------------------
\begin{frame}
\frametitle{\color{red!40!blue}DRP na Windows }
\begin{center}
     \includegraphics[width=0.8\linewidth]{pic/RDP.png}
		\\	{\tiny Zdroj: https://www.howtogeek.com/}   
\end{center}
\end{frame}
%---------------------------------------------------------------------------
\begin{frame}
\frametitle{\color{red!40!blue}Mstsc.exe – RDP client for windows (2019)}
    

    \bi
        \item podívejme se nyní na verzi \textit{Mstsc.exe} z roku 2019
        \item na rozdíl od jiných implementací umožňuje přenášet netriviální data pomocí schránky (clipboard)
        \item pro přenos více souborů používá formát \textit{CF\_HDROP}
            \bi
                \item \textit{"If you are creating a CF\_HDROP format to place in a data object, you will need to construct the file name array."} {\tiny https://docs.microsoft.com/}
                \item Pokud se vytváří objekt \textit{CF\_HDROP}, je mimo jiné nutné specifikovat pole cest k souborům.
                \item jednotlivé cesty k souborům jsou odděleny nulovým charakterem, samotné pole je pak také ukončeno nulovým znakem
                \item \mintinline{python}{c:\folder1.txt'\0'c:\folder2.txt'\0''\0'}
            \ei
    \ei
\end{frame}

%---------------------------------------------------------------------------
\begin{frame}
\frametitle{\color{red!40!blue}Proces kopírování souborů pomocí sdílené schránky}
    \bi
        \item zaměřme se na případ, kdy kopírujeme data ze serveru na klienta
    \ei
    \be
        \item Klient na serveru pomocí RDP zkopíruje data.
        \item Na serveru se vytvoří datový balíček ve formátu \textit{CF\_HDROP}.
        \item Klient na svém PC vloží data (paste).
        \item Vložení spustí sekvenci kroků.
        \item Klient přinutí server připravit a odeslat data na klienta.
        \item Klient aplikace RDP pomocí procesu \textbf{\textit{dpclip.exe}} přistoupí k vytvořenému datovému formátu \textit{CF\_HDROP}. \textbf{\textit{dpclip.exe}} tento formát převede na formát \textit{FileGroupDescriptor}.
        \item Po vytvoření \textit{FileGroupDescriptor} se data pošlou pomocí RDP ke klientovi.
        \item Klient si uloží data do své vlastní schránky a následně data vloží, jako by se jednalo o data z jeho PC.
    \ee
\end{frame} 


%---------------------------------------------------------------------------
\begin{frame}
\frametitle{\color{red!40!blue}Proces kopírování souborů pomocí sdílené schránky}
         \includegraphics[width=1.0\linewidth]{pic/copy_flow.png}
		\\	{\tiny Zdroj: https://research.checkpoint.com/}   
\end{frame}

%---------------------------------------------------------------------------
\begin{frame}
\frametitle{\color{red!40!blue}Zneužití sdílené schránky}
    \bi
        \item ve chvíli, kdy ke klientovi přijdou data \textit{FileGroupDescriptor}, klient nijak nekontroluje jejich správnost a rovnou si je překopíruje do vlastní lokální schránky
        \item zároveň klient nekontroluje, zda je cesta ke kopírovaným souborům škodlivá (neobsahuje relativni cesty)
        \item pokud by se útočníkovi na straně serveru podařilo podvrhnout posílaná data, tak si toho klient nemusí všimnout
        \item útočník by se mohl pokusit do posílaných dat přidat svůj malware
    \ei
    
    \bi
        \item k transformaci dat mezi schránkou na straně serveru a formátem pro přenos dat slouží proces  \textbf{\textit{dpclip.exe}}
        \item tento proces běží jako samostatný proces a může být na přání uživatele ukončen/spuštěn
        \item proto stačí, aby útočník (na straně serveru) ukončil \textbf{\textit{dpclip.exe}} a spustil svůj vlastní, který bude měnit odesílaná data
    \ei
\end{frame}

\begin{frame}
\frametitle{\color{red!40!blue}Zneužití sdílené schránky}
    \bi
        \item k transformaci dat mezi schránkou klienta a formátem pro přenos dat slouží proces \textbf{\textit{dpclip.exe}}
        \item tento proces běží jako samostatný proces a může být na přání uživatele vypnut/spuštěn.
        \item proto stačí, aby útočník ukončil \textbf{\textit{dpclip.exe}} a spustil svůj vlastní, který bude měnit odesílaná data.
    \ei
    \noindent\rule{\textwidth}{0.4pt} \\
    \bi
        \item útočník takto může přidáním souborů do schránky a modifikováním cesty k souboru poslat libovolný soubor do libovolného adresáře na počítač klienta. Například do startup složky. Stačí jen vhodně zvolit cestu k souboru, která se pošle.
    \ei
\end{frame}

\begin{frame}
\frametitle{\color{red!40!blue}Zneužití sdílené schránky}
    \bi
        \item proces vkládání nechtěného souboru na klientův počítač
        \item kopíruje se soubor s názvem \mintinline{python}{..\filename.txt}. Kvůli tomu, že klient nekontroluje, co se mu propíše do schránky, tak útočník může vložit nechtěná data na cizí počítač data, kam on chce. 
    \ei
    \begin{center}
    \includegraphics[width=0.7\linewidth]{pic/demo1.png}
		\\	{\tiny Zdroj: https://research.checkpoint.com/}        
    \end{center}
  
\end{frame}

\begin{frame}
\frametitle{\color{red!40!blue}Zneužití sdílené schránky}
\begin{center}
    \includegraphics[width=0.9\linewidth]{pic/demo3.png}
	\includegraphics[width=0.9\linewidth]{pic/demo2.png}
		\\	{\tiny Zdroj: https://research.checkpoint.com/}
\end{center}
     
\end{frame}

\begin{frame}
\frametitle{\color{red!40!blue}Vyřešení problému}
\begin{center}
    \bi
        \item tento problém byl nahlášen a opraven
        \item společnost Microsoft vydala aktualizaci pro systémy Windows, ve kterých tento problém vyřešila
        \item aktualizace byla vydaná dokonce pro některé již nepodporované systémy (WIN XP...)
    \ei
\end{center}
     
\end{frame}

\begin{frame}
\frametitle{\color{red!40!blue}Zdroje}
\begin{center}
    \bi
        \item \url{https://research.checkpoint.com/2019/reverse-rdp-attack-code-execution-on-rdp-clients/}
        \item \url{https://docs.microsoft.com/en-us/openspecs/windows_protocols/ms-rdpeclip/14e60d52-e0da-4e19-9455-e8643ff17673?redirectedfrom=MSDN}
        \item \url{https://docs.microsoft.com/en-us/windows/win32/shell/clipboard}
        \item \url{https://cyware.com/news/microsoft-patches-reverse-rdp-attacks-and-third-party-clients-ed559adc}
    \ei
\end{center}
     
\end{frame}



\end{document}