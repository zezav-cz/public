\documentclass[aspectration=169]{beamer}
\setbeamertemplate{navigation symbols}{}
%https://mpetroff.net/files/beamer-theme-matrix/


\usepackage[czech]{babel}

\usepackage[T1]{fontenc}
\usepackage[utf8]{inputenc}
\usepackage{times}
\usepackage{hyperref}
\usepackage{caption}
\usepackage{listings}
\usepackage{minted}
\usepackage{ragged2e}
%%\def\lcm{\mathop{\operator@font lcm}}
\newcommand{\nequiv}{\equiv \hspace{-3.5mm} / \hspace{2.5mm} }

\mode<presentation>{
%\usetheme[]{Pittsburgh} % Pittsburgh
\usetheme[]{Pittsburgh} % Pittsburgh
%\usecolortheme{crane}
\useinnertheme{rectangles} %edit theme


%\usetheme{Boadilla}
%\usecolortheme{seahorse}
%\useoutertheme{shadow}
%\usecolortheme{wolverine}
%\usecolortheme{beetle}
}

\def\m#1{{\color{blue} #1}}
%% TZ?
\def\rememberenum{\xdef\rememberenum{\@arabic\c@enumi}}
\def\continueenum{\c@enumi=\rememberenum}
%%%%%%%
%%%%%%%
%%%%%%%%%%%%%%%%%%%%%%%%%%%%%%%%%%%%%%%%%%%
%------------------------------------------------------
%  Následujících 4 řádky upravit
%------------------------------------------------------
\title[Síťová komunikace aplikací v Kubernetes]{Síťová komunikace aplikací v Kubernetes s externími zařízeními v privátní síti}
\author[J. Troják]{Jan Troják}
\date[\today]{} % Kód předmětu, datum, zkráceno "Přednáška číslo."
%------------------------------------------------------
%  Na tuto sekci nesahat
%------------------------------------------------------
\institute[ČVUT FIT]{%
\includegraphics[height=.12\textheight]{../assets/images/LogoCVUT.pdf}\\ \smallskip
České vysoké učení technické v Praze, 
Fakulta informačních technologií\\
Katedra počítačových systémů\\ \smallskip
}
\begin{document}
\begin{frame}
\titlepage
\end{frame}

\begin{frame}{Definice problému a cíl práce}
  \begin{block}{Komunikace interní sítě Kubernetes a přilehlé privátní sítě}
    \begin{columns}[onlytextwidth,T]
      \column{\dimexpr\linewidth-45mm-5mm}
      \begin{itemize}
        \item Kubernetes
        \item Interní Kubernetes síť
        \item Přilehlá privátní síť
        \item Zařízení v přilehlé privátní síti
        \end{itemize}
      \column{45mm}
      \includegraphics[width=45mm]{../assets/images/lab-server.pdf}
    \end{columns}
  \end{block}
  \pause
  \begin{block}{Cíle práce}
  \begin{itemize}
        \item Prozkoumat možnosti síťování v Kubernetes
        \item Umožnit Komunikaci s privátní sítí
        \item TCP a UDP
        \item Integrovat řešení do systému Kubernetes
    \end{itemize}
    \end{block}
\end{frame}
%%%%%%%%%%%%%%%%%%%%%%%%%%%%%%%%%%%%%%%%%%%%%%%%%%%%%%%%%%%%%%%%%%%%%%%%%%%%
\begin{frame}{Motivace práce}
    \begin{itemize}
        \item Testování -- HIL
            \begin{itemize}
                \item Integrace do Cloud
                \item CI/CD
                \item Dynamická konfigurace systému
            \end{itemize}
        \item Quality Assurance
        \item Edge Cloud Computing
            \begin{itemize}
                \item Snížení latence -- přenos dat
                \item Preproccesing dat
            \end{itemize}
        \item Smart Cities
    \end{itemize} 
\end{frame}
%%%%%%%%%%%%%%%%%%%%%%%%%%%%%%%%%%%%%%%%%%%%%%%%%%%%%%%%%%%%%%%%%%%%%%%%%%%%
\begin{frame}{Navrhované řešení}
    \begin{itemize}
        \item KubeEdge -- MQTT, HTTP
    \end{itemize}
    \pause
    \bigskip
    \begin{itemize}
        \item Pod s více síťovými zařízeními
        \item PlumbingGroup -- NetworkAttachmentDeffinition
        \item Proxy
        \item Socat
    \end{itemize}
    \bigskip
    \begin{figure}
        \centering
        \includegraphics[width=0.6\linewidth]{../assets/images/pod.pdf}
    \end{figure}

\end{frame}
%%%%%%%%%%%%%%%%%%%%%%%%%%%%%%%%%%%%%%%%%%%%%%%%%%%%%%%%%%%%%%%%%%%%%%%%%%%%
\begin{frame}{Integrace do Kubernetes}
    \begin{itemize}
    \item Custom Resource Definition
    \begin{itemize}
        \item Device (edge-operator.k8s.dvojak.cz/v1)
        \item Connection (edge-operator.k8s.dvojak.cz/v1)
    \end{itemize}
    \item Operátor EdgeOperator
    \begin{itemize}
        \item ValidationWebHook
        \item Connection Kontroller
        \item Deployment -- Selfhealing
        \item Service -- DNS, Loadbalancing
    \end{itemize}
    \end{itemize}
\end{frame}
%%%%%%%%%%%%%%%%%%%%%%%%%%%%%%%%%%%%%%%%%%%%%%%%%%%%%%%%%%%%%%%%%%%%%%%%%%%%
\begin{frame}{Výsledky}
    \begin{itemize}
        \item github.com/dvojak-cz/Bachelor-Thesis
        \item Instalace -- GitHub Release v1.0.0
        \item Dokumentace: bt.project.dvojak.cz/
    \end{itemize}
    \bigskip
\end{frame}
%%%%%%%%%%%%%%%%%%%%%%%%%%%%%%%%%%%%%%%%%%%%%%%%%%%%%%%%%%%%%%%%%%%%%%%%%%%%
\begin{frame}{Budoucí vývoj}
    \begin{itemize}
        \item Prezentovat výsledky -- Micronova AG (HIL)
        \item Zveřejnit operátor na OperastorHub.io
        \item LinuxDays/InstallFest
    \end{itemize}
\end{frame}
%%%%%%%%%%%%%%%%%%%%%%%%%%%%%%%%%%%%%%%%%%%%%%%%%%%%%%%%%%%%%%%%%%%%%%%%%%%%
\begin{frame}{}\end{frame}
%%%%%%%%%%%%%%%%%%%%%%%%%%%%%%%%%%%%%%%%%%%%%%%%%%%%%%%%%%%%%%%%%%%%%%%%%%%%
\begin{frame}{Otázky k obhajobě}
    \begin{block}{Optimalizace použitelnosti a zobecnění řešení}
    \begin{justify}
        Vámi prezentované řešení je funkční, nicméně statické (pro specifický scénář) a uživatelsky ne příliš přívětivé.  Důsledkem těchto faktů je skutečnost, že Vaše řešení je nesnadno nasaditelné v praxi. Plánujete optimalizaci v tomto směru?
        \end{justify}
    \end{block}
    \bigskip\medskip
    \begin{block}{Podpora protokolů síťové vrstvy (IP\ldots)}
    \begin{justify}
        V případě směrování provozu do jiných sítí a s tímto související nutností překladu adres (NAT) zmiňujete protokoly transportní vrstvy, tj. TCP a UDP, které používají porty. Funguje Vaše řešení čistě i nad protokoly síťové vrstvy jako je např. IP?
    \end{justify}
    \end{block}
\end{frame}
%%%%%%%%%%%%%%%%%%%%%%%%%%%%%%%%%%%%%%%%%%%%%%%%%%%%%%%%%%%%%%%%%%%%%%%%%%%%
\begin{frame}{}\end{frame}
\end{document}
